\documentclass{beamer}
\usepackage{tikz}
\usetikzlibrary{mindmap}
\usetheme{Warsaw} % utilizar un tema pre-diseñado
\usecolortheme{wolverine} % utilizar un esquema de color pre-diseñado

\title{Título de tu Presentación}
\author{Tu nombre}
\date{\today}

\begin{document}

\maketitle

\begin{frame}
\frametitle{Diapositiva 1}

\tikz[mindmap,text=white,
        root concept/.style={concept color=blue},
        level 1 concept/.append style=
{every child/.style={concept color=blue!50}}]
    \node [concept] {Importancia de la Optimización}
    child[grow=120] {node[concept] {Maximizar utilidad}}
    child[grow=90 ] {node[concept] {Minimizar costos}}
    child[grow=60 ] {node[concept] {Tomar decisiones estratégicas}}
    child[grow=30 ] {node[concept] {Mejorar la eficiencia}};

\end{frame}

\begin{frame}
\frametitle{Diapositiva 2}

Aquí va el contenido de tu segunda diapositiva.

\end{frame}

% Continúa con más diapositivas según sea necesario

\end{document}

